\documentclass[a4paper,11pt]{article}
\usepackage[T1]{fontenc}
\usepackage[utf8]{inputenc}
\usepackage{lmodern}

\title{Avaliação de Monografias - Computação Móvel}
\author{Fernando Omar Aluani (NUSP: 6797226)}

\begin{document}

\maketitle

\section{Monografia do Samuel Plaça de Paula}
O título dessa monografia é \textbf{Usando Redes Aleatórias na Análise de Mobilidade}.

\subsection{Razão de Escolha}
Das poucas monografias para escolher (e lembrando do seminário delas), essa foi a que mais me interessou.
Também dei uma breve olhada nas monografias disponíveis e essa foi a que me pareceu mais bem organizada.

\subsection{Qualidade de Legibilidade, Organização e Apresentação do Texto}
A monografia foi muito bem escrita e apresentada. Praticamente não há erros no texto e mesmo sendo
escrito em uma linguagem mais formal, é fácil de entender o que está sendo passado.

A organização da monografia também está ótima, com os devidos assuntos separados em capitulos e
subcapitulos, assim como uma tabela de índices sucinta no ínicio.

\subsection{Qualidade do Conteúdo Técnico do Texto}
Conteúdo técnico da monografia está muito bom. Apresenta e explica bem as partes técnicas do artigo
e da teoria por trás dele.

\subsection{Contribuições do Autor} %monografia apresenta contribuicoes do autor da mesma?
A monografia apresenta contribuições do autor sobre o artigo. Mais notavelmente, ele criou algumas
imagens para explicar alguns pontos do artigo, também reescreveu algumas partes do artigo (como o algoritmo)
e em diversas partes apresentou sua opinião e suposições sobre partes do artigo que não foram propriamente
explicadas.

\subsection{Opinião do Autor sobre o Artigo}
O autor acredita que enquanto a técnica proposta no artigo é boa, sua apresentação no artigo pecou 
em alguns pontos, e que faltar nesses quesitos não invalida a proposta. 

\subsection{Comentários sobre a Monografia}
Capitulo 1, paragrafo 5: tem trechos meio confusos para entender... Poderia refrasear o texto 
para ficar mais simples.

Algoritmo 3.1, linha 5: é adicionado o indice i na urna, sendo que antes no texto ele explica
que a urna tem cópias dos vértices, o que nesse ponto no algoritmo seria Vi.
Depois disso, o algoritmo usa esses valores provenientes da urna, mas não é claro
qual forma é a correta (se era pra eles serem indices ou o próprio vértice)

Capitulo 4.3, paragrafo 1: perto do final do paragrafo, existência escrito como "exitência"

Capitulo 6, ultimo paragrafo: frase "esta publicacao resulta interessante e potencialmente
bastante relevante" é estranha. resulta o q?


\pagebreak

\section{Monografia do Rafael Santos Coelho}
O título dessa monografia é \textbf{Resenha do artigo \textquotedblleft DTN-Based 
Data Aggregation for Timely Information Collection in Disaster Areas\textquotedblright }.

\subsection{Razão de Escolha}
Assim como na primeira monografia que eu avaliei, essa foi uma das que mais me interessou.
Também vi as avaliações dos seminários, e este foi o que obteve mais notas altas entre todas avaliações
de seminários até hoje (18/06).

\subsection{Qualidade de Legibilidade, Organização e Apresentação do Texto}
\subsection{Qualidade do Conteúdo Técnico do Texto}
\subsection{Contribuições do Autor} %monografia apresenta contribuicoes do autor da mesma?
\subsection{Opinião do Autor sobre o Artigo}

\end{document}
