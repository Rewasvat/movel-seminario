\input texbase

\titulo{Avaliação de Monografias}
\materia{MAC0463 - Programação Móvel}

\aluno{Fernando Omar Aluani}{6797226}

\begin{document}
\cabecalho

\section{Monografia do Samuel Plaça de Paula}
O título dessa monografia é \textbf{Usando Redes Aleatórias na Análise de Mobilidade}.\\

\textbf{Nota:} 9.5

\subsection{Razão de Escolha}
Das poucas monografias para escolher (e lembrando do seminário delas) dentre as disponíveis (até hoje, 18/06),
essa foi a que mais me interessou. Também dei uma breve olhada nas monografias disponíveis e essa
foi a que me pareceu mais bem organizada.

\subsection{Qualidade de Legibilidade, Organização e Apresentação do Texto}
A monografia foi muito bem escrita e apresentada. Praticamente não há erros no texto e mesmo sendo
escrito em uma linguagem mais formal, é fácil de entender o que está sendo passado.

A organização da monografia também está ótima, com os devidos assuntos separados em capitulos e
subcapitulos, assim como uma tabela de índices sucinta no ínicio.

\subsection{Qualidade do Conteúdo Técnico do Texto}
Conteúdo técnico da monografia está muito bom. Apresenta e explica bem as partes técnicas do artigo
e da teoria por trás dele.

\subsection{Contribuições do Autor} %monografia apresenta contribuicoes do autor da mesma?
A monografia apresenta várias contribuições do autor sobre o artigo. Mais notavelmente, ele criou algumas
imagens para explicar alguns pontos do artigo, também reescreveu algumas partes do artigo (como o algoritmo)
e em diversas partes apresentou sua opinião e suposições sobre partes do artigo que não foram propriamente
explicadas.

\subsection{Opinião do Autor sobre o Artigo}
O autor acredita que enquanto a técnica proposta no artigo é boa, sua apresentação no artigo pecou 
em alguns pontos, e que faltar nesses quesitos não invalida a proposta. 

\subsection{Comentários sobre a Monografia}
\begin{itemize}
  \item Capítulo 1, parágrafo 5: tem trechos meio confusos para entender... Poderia refrasear o texto 
para ficar mais simples.

  \item Algoritmo 3.1, linha 5: é adicionado o indice i na urna, sendo que antes no texto ele explica
que a urna tem cópias dos vértices, o que nesse ponto no algoritmo seria $v_{i}$.
Depois disso, o algoritmo usa esses valores provenientes da urna, mas não é claro
qual forma é a correta (se era pra eles serem índices ou o próprio vértice)

  \item Capítulo 4.3, parágrafo 1: perto do final do parágrafo, existência está escrito como "exitência"

  \item Capítulo 6, último parágrafo: frase "esta publicação resulta interessante e potencialmente
bastante relevante" é estranha. Resulta o que?
\end{itemize}


\pagebreak

\section{Monografia do Rafael Santos Coelho}
O título dessa monografia é \textbf{Resenha do artigo \textquotedblleft DTN-Based 
Data Aggregation for Timely Information Collection in Disaster Areas\textquotedblright }.\\

\textbf{Nota:} 6.5

\subsection{Razão de Escolha}
Assim como na primeira monografia que eu avaliei, essa foi uma das que mais me interessou dentre as disponíveis.
Também vi as avaliações dos seminários, e este foi o que obteve mais notas altas entre todas avaliações
de seminários até hoje (18/06), e eu queria ver como seria a monografia correspondente.

\subsection{Qualidade de Legibilidade, Organização e Apresentação do Texto}
Legibilidade da monografia está muito boa, não achei nenhum problema nesse quesito. Porém
a organização e apresentação dela pecam em alguns lugares.

A monografia não é muito bem organizada, sendo basicamente texto separado por algumas imagens,
e divididos em alguns capítulos e parágrafos. Há várias partes que se beneficiariam se fossem subdivididas
em subcapítulos, ou organizadas em listas, por exemplo. Também existem algumas figuras que estão
posicionadas em lugares que não condizem com o trecho do texto que a referencia.

\subsection{Qualidade do Conteúdo Técnico do Texto}
O conteúdo técnico do texto é muito bom, com poucas exceções. Notavelmente, algumas partes técnicas
(principalmente as que envolviam trechos com notação matemática) estão meio difíceis de entender,
mas isso pode ser atribuído à qualidade da organização e apresentação do texto.

Também foi notado uma discrepância em um dos 4 trechos de algoritmos mostrados na monografia. Enquanto
é uma pequena discrepância, julgo esse problema importante devido à relevância desses algoritmos para
o assunto tratado na monografia.

\subsection{Contribuições do Autor}
Não há nenhuma. Se o autor fez alguma contribuição, ele não menciona isso na monografia.

No inicio ele diz que é uma resenha do artigo, mas ao longo da monografia, 
aparentemente ele só explica o assunto, mostra os resultados e tal. No capitulo 5,
sobre a Avaliação Experimental, em algumas partes ele fala brevemente sobre o resultado da simulação,
interpretando os gráficos, mas pela simplicidade desses trechos não é possível deduzir se ele
que cunhou essas interpretações do resultado sozinho ou se ele pegou isso do artigo.

\subsection{Opinião do Autor sobre o Artigo}
Só aparece no último capítulo (o 6), Considerações Finais. Onde ele diz que achou a modelagem
do artigo satisfatória, mas que faltava formalismo matemático.

Nesse trecho o autor também nota a dúvida do porque os autores do artigo não fizeram testes
com outros protocolos de roteamento, e registra as perspectivas de trabalhos futuros (pretendidas
pelos autores originais do artigo).

\subsection{Comentários sobre a Monografia}
\begin{itemize}
  \item Figura 3 (Algoritmo DescobertaDaVizinhanca): está uma página antes das Figuras 4, 5 e 6
que mostram as outras partes do algoritmo (e são mostradas em sequência). Essa figura
também está posicionada no meio do capítulo anterior ao capítulo que descreve o Algoritmo (cap4).

  \item Algoritmo Agregação (Figura6), linha 5: devido a essa linha, algoritmo só agrega mensagens
atômicas se distância entre elas for maior que o raio de comunicação (Rcon), enquanto no texto
sobre o problema de agregação (cap3) ele diz que é basicamente o contrário: devido a restrição de 
comunicação duas UMs só podem se comunicar se estiverem a no máximo uma distância Rcon uma da outra.

  \item Capítulo 5, parágrafo 1: é mencionado que nas simulações as UMs se movem tomando rotas ditadas
pelo algoritmo de Dijkstra, mas não é claro como isso funciona.

  \item Capítulo 5, parágrafo 2: frase "a curva indica que o eficiente leva menos tempo para emitir a
mensagem de resposta para o sorvedouro". O 'eficiente'? O que é isso? Acho que queria dizer UM ali, não?
\end{itemize}

\end{document}
