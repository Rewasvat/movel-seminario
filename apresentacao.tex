\documentclass[brazil]{beamer}
\usepackage{beamerthemesplit}
\usepackage[brazilian]{babel}
\usepackage[utf8]{inputenc}
\usepackage{color}
\usepackage{xcolor}
\usepackage{fancybox}
\usepackage{ulem}
\usepackage{upquote}
\usetheme{JuanLesPins}

\title{Seminário - Computação Móvel}
\author{Fernando Omar Aluani}

\begin{document}

%SLIDE 1
\frame{\titlepage}

\section{Pesquisa - Keynote}

%SLIDE 2
\frame{
  \begin{center}
    \LARGE Nanonetworks: A New Frontier in Communications \\
    \normalsize Ian F. Akyildiz, Josep Miquel Jornet, Massimiliano Pierobon
  \end{center}
}

%SLIDE 3
\frame{
  Nanotecnologia consiste no processamento, separação, consolidação e deformação
  de materiais por um átomo ou por uma molécula. \\
  \vspace{7pt}
  Entre os vários objetivos da nanotecnologia, essa pesquisa se foca no
  desenvolvimento de \textbf{nanomáquinas}: Dispositivos funcionais consistindo de
  componentes em nanoescala e capazes de realizar tarefas simples nessa escala.\\
  \vspace{7pt}
  E da inteconexão de nanomáquinas em uma rede, ou \textbf{nanorede}, como uma
  forma de solucionar as limitações de uma nanomáquina individual.
}

%SLIDE 4
\frame{
  As aplicações potenciais de uma nanoredes são imensa, e podem ser classificadas
  em quatro grandes áreas:
  \vspace{10pt}
  \begin{itemize}
    \item Aplicações Biomedicinais.
    \item Aplicações Industriais.
    \item Aplicações Ambientais.
    \item Aplicações Militares.
  \end{itemize}
}

\subsection{Criando Nanomáquinas}

%SLIDE 5
\frame{
  \begin{center}
    \LARGE Manufatura de Nanomáquinas
  \end{center}
}

%SLIDE 6
\frame{
  Capacidades de uma nanomáquina dependem muito de como ela é feita. \\
  \vspace{10pt}
  Classificações de métodos de desenvolvimento de nanomáquinas:\\
  \begin{itemize}
    \item \textit{Top-Down}.
    \item \textit{Bottom-Up}.
    \item Bio-Hibrídas.
  \end{itemize}
}

%SLIDE 7
\frame{
  Componentes \textit{Man-Made}: já foi criado diversos componentes em nano-escala, como
  nanotubos de carbono, transistores de grafeno, etc; \\
  \vspace{10pt}
  Componentes Biológicos: reuso de componentes biológicos como organelas, ATP e DNA,
  enquanto não foram experimentados ainda poderiam ser muito úteis para criação de 
  nanomáquinas com aplicações biomédicas.
}

\subsection{Habilitando Nanocomunicação}

%SLIDE 8
\frame{
  \begin{center}
    \LARGE Habilitando Nanocomunicação
  \end{center}
}

%SLIDE 9
\frame{
  O jeito que nanomáquinas comunicam entre si depende do jeito que foram feitas. \\
  E a aplicação alvo dessas nanomáquinas também limitam o tipo de nanocomunicação 
  que pode ser usado. \\
  \vspace{10pt}
  Métodos propostos vão desde miniaturização de métodos de comunicação existentes
  (baseados em eletromagnetismo, luz, som ou meios mecânicos) à desenvolvimento
  de novos métodos inspirados na biologia.
}

%SLIDE 10
\frame{
  Nanoantenas de grafeno sofrem fenomenos quânticos que a fazem ser mais eficiente
  que antenas normais, mas também fazem a sua eficiência de radiação ser prejudicada.\\
  Nanotubos de carbono foram propostos como a base de um nano-rádio eletromecânico.
  Essa técnica já foi provada para recepção, mas necessitaria de fontes de energia 
  poderosas em nano-escala para transmissão ativa.\\
  \vspace{7pt}
  \textbf{Banda de Terahertz (0.1THz-10Thz)}: frequência de resonância esperada de uma
  nanoantena de grafeno de um $\mu m$ de comprimento. Essa banda tem uma janela de
  transmissão bem larga, suportando altas taxas de transmissão em curtas distâncias
  (Terabits/segundo em menos de 1 metro).
}

%SLIDE 11
\frame{
  Métodos propostos inspirados na biologia: \\
  \vspace{10pt}
  \begin{itemize}
    \item Difusão livre de moléculas;
    \item Feromônios;
    \item Baseados em neurônios (usando fibras nervosas);
    \item Circuitos de fluxos de capilares;
    \item Transporte de moléculas usando bácterias ou nanomotores;
  \end{itemize}
}

\subsection{Protocolos de Nanocomunicação}

%SLIDE 12
\frame{
  \begin{center}
    \LARGE Protocolos de Nanocomunicação
  \end{center}
}

%SLIDE 13
\frame{
  \textbf{Nanoredes de Terahertz}: a banda de Terahertz tem uma 
  largura de banda bem larga. De um lado isso pode ser usado 
  para permitir transmissões rápidas de dados entre nanomáquinas,
  pelo outro lado, isso habilita novas técnicas de acesso de canais,
  o que pode facilitar a tarefa do protocolo MAC.\\
}

%SLIDE 14
\frame{
  \textbf{Nanoredes Moleculares}: no artigo, autor acredita que
  estudo de protocolos vão seguir dois caminhos:\\
  \begin{itemize}
    \item Estruturas e protocolos diretamente inspirados dos processos
    de comunicação achados na natureza;
    \item Paradigmas clássicos adaptados para nanoredes moleculares;
  \end{itemize}
}

\end{document}
