\documentclass[brazil]{beamer}
\usepackage{beamerthemesplit}
\usepackage[brazilian]{babel}
\usepackage[utf8]{inputenc}
\usepackage{color}
\usepackage{xcolor}
\usepackage{fancybox}
\usepackage{ulem}
\usepackage{upquote}
\usetheme{JuanLesPins}

\title{MAC0431 EP2}
\author{Fernando Omar Aluani \\ Jefferson Serafim Ascaneo}

\begin{document}

\frame{\titlepage}

\section{Introdução}

\frame{
  \begin{center}
    \LARGE Introdução
  \end{center}
}

\frame{
  \underline{\Large Propósito:}
  
  \pause
  \vspace{10pt}
  \hspace{10pt}
  Análise da participação de desenvolvedores em um repositório \textit{Git},
  extraindo estatísticas dos commits no projeto.
}

\frame{
  \underline{\Large Fonte dos Dados:}
  
  \pause
  \vspace{10pt}
  \hspace{10pt}
  Relatório dos commits (de qualquer intervalo de tempo) de um repositório.
  Gerado pelo próprio Git pelo comando \\
  \begin{center}
    \textbf{git log --numstat}
  \end{center}
}

\section{Problema}

\frame{
  \begin{center}
    \LARGE Problema
  \end{center}
}

\frame{
  \underline{\Large Entrada:}
  
  \pause
  \hspace{10pt}
  Log dos commits do Git. O log é composto pelas informações de cada commit, do
  mais recente ao mais antigo, de forma sequencial. O ``bloco'' de cada commit
  contêm as seguintes informações:
  
  \pause
  \begin{itemize}
    \item Hashcode que identifica unicamente o commit.
    \item Nome e email do autor do commit.
    \item Data e hora em que o commit foi feito.
    \item Texto descrevendo o commit (escrito pelo autor).
    \item Lista de triplas contendo: número de linhas adicionadas, número de linhas removidas,
          nome do arquivo alterado.
  \end{itemize}
}

\frame{
  \underline{\Large Saída:}
  
  \pause
  \vspace{10pt}
  Para cada desenvolvedor que fez algum commit:
  \begin{itemize}
    \item Número de commits feitos pelo desenvolvedor;
    \item Média, desvio padrão e total de linhas modificadas pelo desenvolvedor;
  \end{itemize}
  
  \pause
  \vspace{10pt}
  Para os commits no geral:
  \begin{itemize}
    \item Média e desvio padrão de linhas modificadas por commit;
  \end{itemize}
  
  \pause
  \vspace{10pt}
  Para cada arquivo alterado (criado, modificado, removido, etc) no repositório:
  \begin{itemize}
    \item Média e desvio padrão de linhas modificadas no arquivo.
  \end{itemize}
}

\section{Abordagem}

\frame{
  \begin{center}
    \LARGE Abordagem
  \end{center}
}

\begin{frame}[fragile]
  \underline{\Large \textbf{Map}:}
  
  \vspace{4pt}
  \begin{itemize}
    \item[Entrada] Par \verb$(Chave, Texto)$.
    \item[Saida] Coleção de pares \verb$(Atributo, Valor)$.
  \end{itemize}
  
  \pause
  \hspace{10pt}
  Onde:
  \begin{itemize}
    \item \verb$Chave$ é o hash identificando o commit;
    \item \verb$Texto$ é o bloco de texto descrevendo o commit;
    \item \verb$Atributo$ é o nome de algum atributo extraído do commit, 
          como por exemplo \verb$numcommits_NomeDoDesenvolvedor$;
    \item \verb$Valor$ é o valor desse atributo extraído do commit;
  \end{itemize}
\end{frame}

\begin{frame}[fragile]
  \underline{\Large \textbf{Reduce}:}
  
  \vspace{4pt}
  \begin{itemize}
    \item[Entrada] Pares produzidos pelo \textbf{Map}.
    \item[Saida] Coleção de pares \verb$(Estatística, Valor)$.
  \end{itemize}
  
  \pause
  \hspace{10pt}
  Onde:
  \begin{itemize}
    \item \verb$Estatística$ é o nome de alguma estatística calculada a partir dos atributos do commit, 
          como por exemplo \verb$NomeDoAtributo_media$;
    \item \verb$Valor$ é o valor da estatística;
  \end{itemize}
\end{frame}


\end{document}
